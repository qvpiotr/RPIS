\documentclass{article}

\usepackage{hyperref}

\title{Analiza zestawu pomiarów kwiatów Irysa}
  \author{\textbf{304358, Piotr WITEK}, czwartek $16^{15}$\\ 
\textit{AGH, Wydział Informatyki Elektroniki i Telekomunikacji}\\
\textit{Rachunek prawdopodobieństwa i statystyka 2020/2021}}
\date{Kraków, \today}


\usepackage{Sweave}
\begin{document}
\input{raport-concordance}
\maketitle

\textit{Ja, niżej podpisany(na) własnoręcznym podpisem deklaruję, że przygotowałem(łam) przedstawiony do oceny projekt samodzielnie i żadna jego część nie jest kopią pracy innej osoby.}
\begin{flushright}
{Piotr Witek}
\end{flushright}

\section{Streszczenie raportu}

 \qquad W ramach projektu na przedmiot Rachunek Prawdopodobieństwa i Statystyka dokonałem analizy danych na podstawie zestawu zawierającego pomiary kwiatów Irysa. Zestaw danych jest stosunkowo niewielki, zawiera 150 pomiarów, 3 rodzajów Irysów.

\vspace{0.2cm}

Korzystając z pakietu R wykonałem podstawową analizę danych. Badałem 4 cechy: Sepal Length, Sepal Width, Petal Length oraz Petal Width. Dokonałem analizy rodzaju rozkładu badanych danych wykorzystując test Shapiro-Wilka. Na podstawie wyników testu, wnioskuję, że rozkład cechy Sepal Width jest zbliżony do rozkładu normalnego. Operacje na pozostałych zmiennych wykluczyły podobieństwo do rozkładu normalnego.

\vspace{0.2cm}

Interesujące stało się badanie korelacji pomiędzy zmiennymi. Korzystając z operacji, które udostępnia środowisko do obliczeń statystycznych stworzyłem macierz korelacji, z której odczytałem, jakie zmienne korelują z innymi. Stworzyłem serię wykresów obrazujących te zależności z użyciem regresji, opisując jaka zmienna zależy od innej i w jakim stopniu.

\vspace{0.2cm}

Aby podkreślić czytelność i przejrzystość wykonywanej analizy, korzystałem z wielu tabel i wykresów.

\vspace{0.2cm}

Badany przeze mnie zestaw danych jest często wykorzystywany do nauki Machine Learningu, ze względu na przejrzystość danych, stosunkowo niewielką liczbę zmiennych, oraz ciekawe zależności między zmiennymi.


\section{Opis danych}

Dataset 'Iris spiecies' pochodzi ze strony \url{https://www.kaggle.com/uciml/iris} . Ten zbiór danych został opisany w 1936 roku i zawiera zmierzone wartości parametrów kwiatów trzech gatunków Irysów: Iris Setosa, Iris Versicolour i Iris Virginica.


\vspace{0.2cm}
Nazwy parametrów:

\begin{Schunk}
\begin{Sinput}
> names(data)
\end{Sinput}
\begin{Soutput}
[1] "Id"            "SepalLengthCm" "SepalWidthCm"  "PetalLengthCm"
[5] "PetalWidthCm"  "Species"      
\end{Soutput}
\end{Schunk}


Podsumowanie datasetu:

\begin{Schunk}
\begin{Sinput}
> summary(data)
\end{Sinput}
\begin{Soutput}
       Id         SepalLengthCm    SepalWidthCm   PetalLengthCm  
 Min.   :  1.00   Min.   :4.300   Min.   :2.000   Min.   :1.000  
 1st Qu.: 38.25   1st Qu.:5.100   1st Qu.:2.800   1st Qu.:1.600  
 Median : 75.50   Median :5.800   Median :3.000   Median :4.350  
 Mean   : 75.50   Mean   :5.843   Mean   :3.054   Mean   :3.759  
 3rd Qu.:112.75   3rd Qu.:6.400   3rd Qu.:3.300   3rd Qu.:5.100  
 Max.   :150.00   Max.   :7.900   Max.   :4.400   Max.   :6.900  
  PetalWidthCm     Species         
 Min.   :0.100   Length:150        
 1st Qu.:0.300   Class :character  
 Median :1.300   Mode  :character  
 Mean   :1.199                     
 3rd Qu.:1.800                     
 Max.   :2.500                     
\end{Soutput}
\end{Schunk}

\section{Potrzebne biblioteki użyte w projekcie}
\begin{Schunk}
\begin{Sinput}
> library(ggplot2)
> library(rstudioapi)
> library(gridExtra)
> library(grid)
> library(plyr)
> library(GGally)
> library(plotly)
> library(e1071)
\end{Sinput}
\end{Schunk}

\newpage

\section{Podstawowa analiza danych}

W ramach podstawowej analizy danych dla każdej z czterech zmiennych (Sepal Length, Sepal Width, Petal Length, Petal Width) obliczono wartości takie jak: odchylenie standardowe, rozstęp, rozstęp kwartlowy, skośność, momenty oraz wykonano histogram.

\subsection {Sepal Length}

Podsumowanie:

\begin{Schunk}
\begin{Sinput}
> summary(data$SepalLength)
\end{Sinput}
\begin{Soutput}
   Min. 1st Qu.  Median    Mean 3rd Qu.    Max. 
  4.300   5.100   5.800   5.843   6.400   7.900 
\end{Soutput}
\end{Schunk}
Odchylenie standardowe:
\begin{Schunk}
\begin{Sinput}
> sd(data$SepalLength)
\end{Sinput}
\begin{Soutput}
[1] 0.8280661
\end{Soutput}
\end{Schunk}
Rozstęp:
\begin{Schunk}
\begin{Sinput}
> range(data$SepalLength)
\end{Sinput}
\begin{Soutput}
[1] 4.3 7.9
\end{Soutput}
\end{Schunk}
Rozstęp kwartylowy:
\begin{Schunk}
\begin{Sinput}
> IQR(data$SepalLength)
\end{Sinput}
\begin{Soutput}
[1] 1.3
\end{Soutput}
\end{Schunk}
Skośność:
\begin{Schunk}
\begin{Sinput}
> skewness(data$SepalLength)
\end{Sinput}
\begin{Soutput}
[1] 0.3086407
\end{Soutput}
\end{Schunk}
Momenty centralne: 
\begin{Schunk}
\begin{Sinput}
> moment(data$SepalLength,0.25)
\end{Sinput}
\begin{Soutput}
[1] 1.55188
\end{Soutput}
\begin{Sinput}
> moment(data$SepalLength,0.5)
\end{Sinput}
\begin{Soutput}
[1] 2.411318
\end{Soutput}
\begin{Sinput}
> moment(data$SepalLength,0.75)
\end{Sinput}
\begin{Soutput}
[1] 3.75136
\end{Soutput}
\begin{Sinput}
> moment(data$SepalLength,1)
\end{Sinput}
\begin{Soutput}
[1] 5.843333
\end{Soutput}
\end{Schunk}

Histogram:
\begin{Schunk}
\begin{Sinput}
> ggplot(data, aes(x=SepalLengthCm, fill=factor(Species)))+
+   geom_bar()+ 
+   scale_fill_discrete(name="Species")
\end{Sinput}
\end{Schunk}
\includegraphics{raport-011}


\subsection {SepalWidth}

Podsumowanie:

\begin{Schunk}
\begin{Sinput}
> summary(data$SepalWidth)
\end{Sinput}
\begin{Soutput}
   Min. 1st Qu.  Median    Mean 3rd Qu.    Max. 
  2.000   2.800   3.000   3.054   3.300   4.400 
\end{Soutput}
\end{Schunk}
Odchylenie standardowe:
\begin{Schunk}
\begin{Sinput}
> sd(data$SepalWidth)
\end{Sinput}
\begin{Soutput}
[1] 0.4335943
\end{Soutput}
\end{Schunk}
Rozstęp:
\begin{Schunk}
\begin{Sinput}
> range(data$SepalWidth)
\end{Sinput}
\begin{Soutput}
[1] 2.0 4.4
\end{Soutput}
\end{Schunk}
Rozstęp kwartylowy:
\begin{Schunk}
\begin{Sinput}
> IQR(data$SepalWidth)
\end{Sinput}
\begin{Soutput}
[1] 0.5
\end{Soutput}
\end{Schunk}
Skośność:
\begin{Schunk}
\begin{Sinput}
> skewness(data$SepalWidth)
\end{Sinput}
\begin{Soutput}
[1] 0.3274013
\end{Soutput}
\end{Schunk}
Momenty centralne: 
\begin{Schunk}
\begin{Sinput}
> moment(data$SepalWidth,0.25)
\end{Sinput}
\begin{Soutput}
[1] 1.319481
\end{Soutput}
\begin{Sinput}
> moment(data$SepalWidth,0.5)
\end{Sinput}
\begin{Soutput}
[1] 1.743213
\end{Soutput}
\begin{Sinput}
> moment(data$SepalWidth,0.75)
\end{Sinput}
\begin{Soutput}
[1] 2.305896
\end{Soutput}
\begin{Sinput}
> moment(data$SepalWidth,1)
\end{Sinput}
\begin{Soutput}
[1] 3.054
\end{Soutput}
\end{Schunk}

Histogram:
\begin{Schunk}
\begin{Sinput}
> ggplot(data, aes(x=SepalWidthCm, fill=factor(Species)))+
+   geom_bar()+ 
+   scale_fill_discrete(name="Species")
\end{Sinput}
\end{Schunk}
\includegraphics{raport-018}

\subsection {Petal length}

Podsumowanie:

\begin{Schunk}
\begin{Sinput}
> summary(data$PetalLength)
\end{Sinput}
\begin{Soutput}
   Min. 1st Qu.  Median    Mean 3rd Qu.    Max. 
  1.000   1.600   4.350   3.759   5.100   6.900 
\end{Soutput}
\end{Schunk}
Odchylenie standardowe:
\begin{Schunk}
\begin{Sinput}
> sd(data$PetalLength)
\end{Sinput}
\begin{Soutput}
[1] 1.76442
\end{Soutput}
\end{Schunk}
Rozstęp:
\begin{Schunk}
\begin{Sinput}
> range(data$PetalLength)
\end{Sinput}
\begin{Soutput}
[1] 1.0 6.9
\end{Soutput}
\end{Schunk}
Rozstęp kwartylowy:
\begin{Schunk}
\begin{Sinput}
> IQR(data$PetalLength)
\end{Sinput}
\begin{Soutput}
[1] 3.5
\end{Soutput}
\end{Schunk}
Skośność:
\begin{Schunk}
\begin{Sinput}
> skewness(data$PetalLength)
\end{Sinput}
\begin{Soutput}
[1] -0.2689994
\end{Soutput}
\end{Schunk}
Momenty centralne: 
\begin{Schunk}
\begin{Sinput}
> moment(data$PetalLength,0.25)
\end{Sinput}
\begin{Soutput}
[1] 1.355716
\end{Soutput}
\begin{Sinput}
> moment(data$PetalLength,0.5)
\end{Sinput}
\begin{Soutput}
[1] 1.874027
\end{Soutput}
\begin{Sinput}
> moment(data$PetalLength,0.75)
\end{Sinput}
\begin{Soutput}
[1] 2.634887
\end{Soutput}
\begin{Sinput}
> moment(data$PetalLength,1)
\end{Sinput}
\begin{Soutput}
[1] 3.758667
\end{Soutput}
\end{Schunk}

Histogram:
\begin{Schunk}
\begin{Sinput}
> ggplot(data, aes(x=PetalLengthCm, fill=factor(Species)))+
+   geom_bar()+ 
+   scale_fill_discrete(name="Species")
\end{Sinput}
\end{Schunk}
\includegraphics{raport-025}

\subsection {Petal Width}

Podsumowanie:

\begin{Schunk}
\begin{Sinput}
> summary(data$PetalWidth)
\end{Sinput}
\begin{Soutput}
   Min. 1st Qu.  Median    Mean 3rd Qu.    Max. 
  0.100   0.300   1.300   1.199   1.800   2.500 
\end{Soutput}
\end{Schunk}
Odchylenie standardowe:
\begin{Schunk}
\begin{Sinput}
> sd(data$PetalWidth)
\end{Sinput}
\begin{Soutput}
[1] 0.7631607
\end{Soutput}
\end{Schunk}
Rozstęp:
\begin{Schunk}
\begin{Sinput}
> range(data$PetalWidth)
\end{Sinput}
\begin{Soutput}
[1] 0.1 2.5
\end{Soutput}
\end{Schunk}
Rozstęp kwartylowy:
\begin{Schunk}
\begin{Sinput}
> IQR(data$PetalWidth)
\end{Sinput}
\begin{Soutput}
[1] 1.5
\end{Soutput}
\end{Schunk}
Skośność:
\begin{Schunk}
\begin{Sinput}
> skewness(data$PetalWidth)
\end{Sinput}
\begin{Soutput}
[1] -0.102906
\end{Soutput}
\end{Schunk}
Momenty centralne: 
\begin{Schunk}
\begin{Sinput}
> moment(data$PetalWidth,0.25)
\end{Sinput}
\begin{Soutput}
[1] 0.9843897
\end{Soutput}
\begin{Sinput}
> moment(data$PetalWidth,0.5)
\end{Sinput}
\begin{Soutput}
[1] 1.017222
\end{Soutput}
\begin{Sinput}
> moment(data$PetalWidth,0.75)
\end{Sinput}
\begin{Soutput}
[1] 1.08996
\end{Soutput}
\begin{Sinput}
> moment(data$PetalWidth,1)
\end{Sinput}
\begin{Soutput}
[1] 1.198667
\end{Soutput}
\end{Schunk}

Histogram:
\begin{Schunk}
\begin{Sinput}
> ggplot(data, aes(x=PetalWidthCm, fill=factor(Species)))+
+   geom_bar()+ 
+   scale_fill_discrete(name="Species")
\end{Sinput}
\end{Schunk}
\includegraphics{raport-032}

\newpage

\section{Analiza danych przy użyciu histogramu gęstości}

Uwtorzono histogramy obrazujące gęstości każdego atrybutu z zaznaczenie podziału na rodzaje irysów. Dzięki tym wykresom możemy przewidywać rozkład dla każdego parametru oraz jesteśmy w stanie zauwazyć rozdział rodzajów irysów.

\subsection{Sepal Length}
\begin{Schunk}
\begin{Sinput}
> DhistSl <- ggplot(data, aes(x=SepalLengthCm, colour=Species, fill=Species)) +
+   geom_density(alpha=.3) +
+   geom_vline(aes(xintercept=mean(SepalLengthCm),  colour=Species),linetype="dashed", color="grey", size=1)+
+   xlab("Sepal Length") +  
+   ylab("Density")
> DhistSl
\end{Sinput}
\end{Schunk}
\includegraphics{raport-033}

\newpage

\subsection{Sepal Width}

\begin{Schunk}
\begin{Sinput}
> DhistSw <- ggplot(data, aes(x=SepalWidthCm, colour=Species, fill=Species)) +
+   geom_density(alpha=.3) +
+   geom_vline(aes(xintercept=mean(SepalWidthCm),  colour=Species),linetype="dashed", color="grey", size=1)+
+   xlab("Sepal Width") +  
+   ylab("Density") 
> DhistSw
\end{Sinput}
\end{Schunk}
\includegraphics{raport-034}
\newpage
\subsection{Petal Length}

\begin{Schunk}
\begin{Sinput}
> DhistPl <- ggplot(data, aes(x=PetalLengthCm, colour=Species, fill=Species)) +
+   geom_density(alpha=.3) +
+   geom_vline(aes(xintercept=mean(PetalLengthCm),  colour=Species),linetype="dashed", color="grey", size=1)+
+   xlab("Petal Length") +  
+   ylab("Density") 
> DhistPl
\end{Sinput}
\end{Schunk}
\includegraphics{raport-035}
\newpage
\subsection{Petal Width}

\begin{Schunk}
\begin{Sinput}
> DhistPw <- ggplot(data, aes(x=PetalWidthCm, colour=Species, fill=Species)) +
+   geom_density(alpha=.3) +
+   geom_vline(aes(xintercept=mean(PetalWidthCm),  colour=Species),linetype="dashed", color="grey", size=1)+
+   xlab("Petal Width") +  
+   ylab("Density") 
> DhistPw
\end{Sinput}
\end{Schunk}
\includegraphics{raport-036}

\subsection{Wnioski}
Na podstawie analizy wykresów gęstości można podejrzewać, że rozkłąd zmiennej Sepal Width jest zblizony do rozkładu normalnego.

\newpage
\section{Analiza z użyciem wykresów pudełkowych}

Analizę przeprowadzono w celu wykrycia wartości odstających(tzw. outliers) oraz w celu zobrazowania pokrycia się różnych klas irysów. Wykresy pudełkowe stanowią dobre narzedzie do ilustrowania różnic pomiędzy porównywanymi grupami.

\subsection{Sepal Length}
\begin{Schunk}
\begin{Sinput}
> ggplot(data, aes(Species, SepalLengthCm, fill=Species)) + 
+   geom_boxplot()+
+   scale_y_continuous("Sepal Length)", breaks= seq(0,30, by=.5))+
+   labs(title = "Sepal Length", x = "Species")
\end{Sinput}
\end{Schunk}
\includegraphics{raport-037}
\newpage

\subsection{Sepal Width}
\begin{Schunk}
\begin{Sinput}
> ggplot(data, aes(Species, SepalWidthCm, fill=Species)) + 
+   geom_boxplot()+
+   scale_y_continuous("Sepal Width)", breaks= seq(0,30, by=.5))+
+   labs(title = "Sepal Width", x = "Species")
\end{Sinput}
\end{Schunk}
\includegraphics{raport-038}
\newpage

\subsection{Petal Length}
\begin{Schunk}
\begin{Sinput}
> ggplot(data, aes(Species, PetalLengthCm, fill=Species)) + 
+   geom_boxplot()+
+   scale_y_continuous("Petal Length)", breaks= seq(0,30, by=.5))+
+   labs(title = "Petal Length", x = "Species")
\end{Sinput}
\end{Schunk}
\includegraphics{raport-039}
\newpage

\subsection{Petal Width}
\begin{Schunk}
\begin{Sinput}
> ggplot(data, aes(Species, PetalWidthCm, fill=Species)) + 
+   geom_boxplot()+
+   scale_y_continuous("Petal Width)", breaks= seq(0,30, by=.5))+
+   labs(title = "Petal Width", x = "Species")
\end{Sinput}
\end{Schunk}
\includegraphics{raport-040}
\subsection{Wnioski}

Przy niektórych "pudełkach" można zauważyć pojedyncze wartości odstające. Dla zmiennych PetalWidth i PetalLength boxy nie pokrywają się, a dla zmiennym SepalWidth i SepalLength pokrywają się.


\newpage

\section{Analiza rodzaju rozkładu badanych cech}
Do analizy użyto testu Shapiro-Wilk z uwagi na stosunkowo niewielki rozmiar danych. W ramach testu rozkładu wykorzystano paramentr jakim jest kurtoza. Kurtoza rozkładu normalnego wynosi 0.

\subsection{Sepal Length}
\begin{Schunk}
\begin{Sinput}
> kurtosis(data$SepalLengthCm)
\end{Sinput}
\begin{Soutput}
[1] -0.6058125
\end{Soutput}
\end{Schunk}

\begin{Schunk}
\begin{Sinput}
> shapiro.test(data$SepalLengthCm)
\end{Sinput}
\begin{Soutput}
	Shapiro-Wilk normality test

data:  data$SepalLengthCm
W = 0.97609, p-value = 0.01018
\end{Soutput}
\end{Schunk}
\subsubsection{Wnioski}
Kurtoza jest ujemna, wartości cechy mniej skoncentrowane niż przy rozkładzie normalnym. p-value < 0.05, rozkład nie jest normalny.

\subsection{Sepal Width}
\begin{Schunk}
\begin{Sinput}
> kurtosis(data$SepalWidthCm)
\end{Sinput}
\begin{Soutput}
[1] 0.1983681
\end{Soutput}
\end{Schunk}

\begin{Schunk}
\begin{Sinput}
> shapiro.test(data$SepalWidthCm)
\end{Sinput}
\begin{Soutput}
	Shapiro-Wilk normality test

data:  data$SepalWidthCm
W = 0.98379, p-value = 0.07518
\end{Soutput}
\end{Schunk}

\subsubsection{Wnioski}
Kurtoza jest dodatnia, lecz bliska 0, wartości cechy bardziej skoncentrowane niż przy rozkładzie normalnym. p-value > 0.05, przyjmujemy hipotezę, że rozkład jest zbliżony do normalnego.

\newpage

\subsection{Petal Length}
\begin{Schunk}
\begin{Sinput}
> kurtosis(data$PetalLengthCm)
\end{Sinput}
\begin{Soutput}
[1] -1.416683
\end{Soutput}
\end{Schunk}

\begin{Schunk}
\begin{Sinput}
> shapiro.test(data$PetalLengthCm)
\end{Sinput}
\begin{Soutput}
	Shapiro-Wilk normality test

data:  data$PetalLengthCm
W = 0.87642, p-value = 7.545e-10
\end{Soutput}
\end{Schunk}

\subsubsection{Wnioski}
Kurtoza jest ujemna, wartości cechy mniej skoncentrowane niż przy rozkładzie normalnym. p-value < 0.05, rozkład nie jest normalny.

\subsection{Petal Width}
\begin{Schunk}
\begin{Sinput}
> kurtosis(data$PetalWidthCm)
\end{Sinput}
\begin{Soutput}
[1] -1.357368
\end{Soutput}
\end{Schunk}

\begin{Schunk}
\begin{Sinput}
> shapiro.test(data$PetalWidthCm)
\end{Sinput}
\begin{Soutput}
	Shapiro-Wilk normality test

data:  data$PetalWidthCm
W = 0.90262, p-value = 1.865e-08
\end{Soutput}
\end{Schunk}

\subsubsection{Wnioski}
Kurtoza jest ujemna, wartości cechy mniej skoncentrowane niż przy rozkładzie normalnym. p-value < 0.05, rozkład nie jest normalny.

\newpage

\section{Korelacja}
Macierz korelacji w datasecie:
\begin{Schunk}
\begin{Sinput}
> ggpairs(data = data,
+         title = "Correlation Plot",
+         upper = list(continuous = wrap("cor", size = 5)), 
+         lower = list(continuous = "smooth")
+ )
\end{Sinput}
\end{Schunk}
\includegraphics{raport-049}
\begin{Schunk}
\begin{Sinput}
> cor(data[, 2:5])
\end{Sinput}
\begin{Soutput}
              SepalLengthCm SepalWidthCm PetalLengthCm PetalWidthCm
SepalLengthCm     1.0000000   -0.1093692     0.8717542    0.8179536
SepalWidthCm     -0.1093692    1.0000000    -0.4205161   -0.3565441
PetalLengthCm     0.8717542   -0.4205161     1.0000000    0.9627571
PetalWidthCm      0.8179536   -0.3565441     0.9627571    1.0000000
\end{Soutput}
\end{Schunk}
\newpage
Zauważam, że istnieje wysoka korelacja pomiędzy PetalWidth i PetalLength - 96\%

\begin{Schunk}
\begin{Sinput}
> ggplot(data, aes(x=PetalWidthCm, y=PetalLengthCm))+
+   geom_point(aes(colour=Species))+
+   geom_smooth(method='lm')+
+   ggtitle("Correlation between Petal Length and Petal Width")
\end{Sinput}
\end{Schunk}
\includegraphics{raport-051}

\newpage

Zauważam również silną korelację pomiędzy SepalLength i PetalLength - 87\%
\begin{Schunk}
\begin{Sinput}
> ggplot(data, aes(x=SepalLengthCm, y=PetalLengthCm))+
+   geom_point(aes(colour=Species))+
+   geom_smooth(method='lm')+
+   ggtitle("Correlation between Petal Length and Sepal Length")
\end{Sinput}
\end{Schunk}
\includegraphics{raport-052}
\newpage
Korelacja pomiędzy SepalLength i PetalWidth wynosi 82\%
\begin{Schunk}
\begin{Sinput}
> ggplot(data, aes(x=SepalLengthCm, y=PetalWidthCm))+
+   geom_point(aes(colour=Species))+
+   geom_smooth(method='lm')+
+   ggtitle("Correlation between Petal Width and Sepal Length")
\end{Sinput}
\end{Schunk}
\includegraphics{raport-053}
\newpage
Korelacja pomiędzy wartościami SepalLength i SepalWidth jest ujemna i wynosi -0.1. Ujemna korelacja oznacza, że wraz ze wzrostem/ spadkiem jednej zmiennej, druga zmnienna zachowuje się odwrotnie.
\begin{Schunk}
\begin{Sinput}
> ggplot(data, aes(x=SepalWidthCm, y=SepalLengthCm))+
+   geom_point(aes(colour=Species))+
+   geom_smooth(method='lm')+
+   ggtitle("Correlation between Sepal Length and Sepal Width")
\end{Sinput}
\end{Schunk}
\includegraphics{raport-054}




\end{document}


